\documentclass{article}
\usepackage{fixltx2e}
\usepackage{listings}
\usepackage{graphicx}
%\usepackage{minted}
%\usepackage{upquote}
\usepackage[margin=0.75in]{geometry}
\usepackage{url}
\usepackage[T1]{fontenc}

\AtBeginDocument{%
  \def\PYZsq{\textquotesingle}%
}

\title{CS 246 Fall 2016 - Tutorial 9}
\author{}
\date{November 9, 2016}
\begin{document}
\maketitle

\section{Summary}
\begin{itemize}
\item Abstract Classes
\item Vectors
\item Exception Handling
\item Code Review
\end{itemize}

\section{Tip of the Week: commands from vi}

\begin{itemize}
    \item You can do find and replace with vim using: :\%s/old/new/g where \texttt{old} is the text you want to replace, and \texttt{new} is the replacement text.

    \item Similarly, you can replace with a confirmation using: :\%s/old/new/gc
\end{itemize}


\section{Abstract Classes}

\begin{itemize}

\item The purpose of an abstract class is to allow subclasses to inherit from a base class containing information that is common to all other subclasses, and when you know there shouldn't be any instance of the base class. A class becomes 'abstract' when it has one or more pure virtual functions.

\item Note: Your destructors should always be virtual. They must always have an implementation; even if they are pure virtual.

\item Why? Because the destructor of the base class is called even when a derived class is destroyed; this is because every derived class possesses the components of the base class. Thus, the destructor of the base must have some implementation (even if the implementation is empty).

\end{itemize}

Note: Any function that is pure virtual requires its own implementation in the subclasses. We declare a function as pure virtual when we add \texttt{= 0;} to the end of it.

\begin{verbatim}



class B{
  ...
  virtual string hello() = 0; // the function hello is pure virtual
                              // thus making B an abstract class.
  virtual ~B(){};
};
\end{verbatim}
\newpage
\begin{verbatim}
class A: public B{
    ... // inherits its fields from B
    A* arr;
    A(): arr{new A[5]} {}
    ~A(){delete [] arr;}
    string hello() override{
        return "bonjour";
    }
};

\end{verbatim}

Class A inherits from the abstract class B which has a pure virtual destructor. Now, every class that inherits from B has to provide its own implementation of the destructor.

\section{Vectors}

Vectors make dynamic allocation easier; any time you make an array, you could probably achieve the same result with a vector.

Example:
\begin{verbatim}
#include <vector>
#include <iostream>

using namespace std;
int main(){
    vector<int> arr; //creating a vector of integers
    int x;
    while(cin >> x){
        arr.emplace_back(x); //continuously adding integers to arr
                          //and increasing the size of the array without
                          //manually allocating more memory.
    }
    for(int i = 0; i < arr.size(); i++){
        cout << arr[i]; //outputting the values in arr.
    }
}

\end{verbatim}

The vector class has a wealth of functions to access and manipulate the elements of the vector; knowing them is a matter of looking up the functions in the vector class and how to use them.

\section{Exception Handling}

Instead of using C-style methods of handling errors, we use exceptions in C++ to control the behaviour of the program when errors arise.

Recall that we catch exceptions with try and catch blocks.

\begin{verbatim}
    try{
        throw 42;
    }
    catch(...) // ... means 'catch anything'
    {
        cerr << "Caught something" << endl;
    }
\end{verbatim}
\newpage
We can throw \textbf{anything}, and we always catch by reference.

Consider the code below:

\begin{verbatim}
class BadInput{};

class BadNumber: public BadInput{
    string what() { return "no number given" }
};

int main(){
    try{
        cin >> x;
        if (x < 50){
            throw BadNumber{};
        }
    }
    catch(BadInput& b){}
    catch(BadNumber& b){ cerr << b.what() << endl; } //accessing auxilliary information
                                                     //from object that was caught

}
\end{verbatim}
\begin{itemize}
\item Question: Which handler would run?

\item Note: If we weren't passing objects by reference in the catch parameter, slicing could occur; potentially resulting in missing information.

\item Good practice: Throw by value, catch by reference.

\end{itemize}


\end{document}
